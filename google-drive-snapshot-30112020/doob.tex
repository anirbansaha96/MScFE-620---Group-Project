\documentclass{article}
\usepackage[utf8]{inputenc}

\title{Doob part}
\author{plantvsbird }
\date{November 2020}

\begin{document}

\maketitle


% Submit a report illustrating the early history of probability theory from a measure- theoretic perspective. Your discussion must address the following:


% 1. Kolmogorov and his axioms of probability
% 2. Markov and Markov processes
% 3. J. L. Doob and the development of martingales

\section{Introduction}


Doob's first paper in probability was Probability and Statistics. The paper inherited Kolmogoroff's method, using tools of mathematical formulations and more specifically the measure theory to describe and build the theory of probability, and to provide a mathematical basis for study of statistics. In paper, he showed the last point by proving validity of maximum likelihood, one of the classic statistics method.

In the same year, Doob's another paper Stochastic Processes and Statistics which builds on top of Khintchine's investigation of stochastic process. He developed two theorems of stochastic process which can also be used to prove the validity of the method of maximum likelihood.

Later in Probability as Measure, Doob made it clear that he regards the probability theory as a branch of measure theory, with clear, defined, and mathematical methods, a study which is separate from actual experiments. He also hopes that with a strong mathematical foundation, his work can be applied to all seemingly simple problems and be useful for all statisticians. 

With the groundwork done, Doob introduced Martingale theory. The term martingale comes from a gambling practice where gamblers always bet the entire amount lost, so that each win will make the gambler even.

In What is A Martingale, Doob gives a more concise account, compared to his own book Stochastic Processes, on the subject:

With sample space $\Omega$, the expectation of a random variable $x$ can be defined as $E[x]=\Sigma_j x(\omega_j)p_j$, where $\omega_j \in \Omega$, $x(\omega_j)$ is the value of random variable $x$ with outcome $omega_j$, and $p_j$ refers to probability of the outcome. Without using the term "filtration" explicitly, Doob defines $\mathcal{F}_1 \subset \mathcal{F}_2 \subset ...$ a finite or infinite increasing sequence of $\sigma$-algebras generated by partitions of $\Omega$. The sequence of $\{x_n, n \geq 1\}$ is called martingale relative to $\{\mathcal{F}_n, n \geq 1\}$ if
\begin{equation}
	E\{x_n | \mathcal{F}_m\}= x_m
\end{equation}
for $m < n$, and each $x_n$ has an expectation. This simplistic definition can also be expanded to an uncountable sample space.

In a more concrete example, if we define $x_1$ as a gambler's fortune before they play a game, and $x_2$ as the fortune after the game, the game can be considered fair if
\begin{equation}
	E\{x_2\} = x_1
\end{equation}

and with more game played, if the fairness of the game still holds, we have

\begin{equation}
	E\{x_{n+1} | x_1, ..., x_n\} = x_n, n=1,2,...
\end{equation}

And Doob defines a mathematical model of fair game as a martingale

\begin{equation}
\{x_n, \mathcal{F}_n, n \geq 1 \}
\end{equation}

relative to some stated Borel fields. The field $\mathcal{F}_n$ can summarize all the influence variables up to time $n$.

As Doob explained in What is Martingale and Stochastic Processes, the concept of martingale is useful in statistics and information theory because it formally models the concept of the game being "fair" when it allows player to learn more and more information as they play. 

\section{Work Cited}
Joseph L Doob and Development of Probability Theory https://www.ias.ac.in/article/fulltext/reso/020/04/0286-0288


J. L. Doob. “Probability and Statistics.” Transactions of the American Mathematical Society, vol. 36, no. 4, Oct. 1934, pp. 759–775. EBSCOhost, doi:10.2307/1989822.

Doob, J. L. “Probability as Measure.” The Annals of Mathematical Statistics, vol. 12, no. 2, 1941, pp. 206–214. JSTOR, www.jstor.org/stable/2235768. Accessed 26 Nov. 2020.

Doob, J. L. “What Is a Martingale?” The American Mathematical Monthly 78, no. 5 (May 1971): 451–63. https://doi.org/10.1080/00029890.1971.11992788.


Doob, Joseph L. Stochastic Processes . New York: Wiley, 1991. Print.



\end{document}
