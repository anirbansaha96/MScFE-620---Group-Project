\documentclass{article}
\usepackage[utf8]{inputenc}
\usepackage[margin=2.5cm]{geometry}
\usepackage{color,soul}
\usepackage[style=authoryear]{biblatex}
\usepackage{amsmath,amssymb}
\usepackage{setspace}
\usepackage{csquotes}

% following package makes url's in references look nicer
\usepackage{url}
%% Define a new 'leo' style for the package that will use a smaller font.
\makeatletter
\def\url@leostyle{%
  \@ifundefined{selectfont}{\def\UrlFont{\sf}}{\def\UrlFont{\small\ttfamily}}}
\makeatother
%% Now actually use the newly defined style.
\urlstyle{leo}
\onehalfspacing
\addbibresource{references.bib}
\title{Trinomial Trees \\ MScFE 620 Group Project Submission 2}

\date{\today}

\author{
  Avhad, Prajakta\\
  \texttt{prajakta.s.avhad@gmail.com}
  \and
  Hugo, Pepar Thomas Jay\\
  \texttt{peparhugo@gmail.com}
  \and
  Lin, Hong\\
  \texttt{plantvsbird@gmail.com}
  \and
  Ramsay, Ben\\
  \texttt{ramsay.ben@gmail.com}
  \and
  Wang, Jiao Yu\\
  \texttt{joseph\_wang@live.ie}
}


\begin{document}
\maketitle  

%%%%%%%%%%%%%%%%%%%%%%%%%%%%%%%%%%
%%%%%%%%%%  TO HERE  %%%%%%%%%%%%%
%%%%%%%%%%%%%%%%%%%%%%%%%%%%%%%%%%

\section{Part A - Construction of Trinomial Tree}

\hl{Pepar's code goes here.}
This is the other extreme where $\beta$, $\rho$, $v$ are 0. 

\begin{align}
    \alpha+\gamma+\theta+\eta+\psi+\lambda & = 1 \label{eq:EMM-cond-1} \\
    2\alpha+2\gamma + \eta - 2\psi - 2\lambda & = 1 \label{eq:EMM-cond-2}\\
    6\alpha - 6\gamma + 4\theta -4\eta -10\psi + 10\lambda & = 1 \label{eq:EMM-cond-3}\\
    \alpha & = 2\gamma \label{eq:EMM-cond-4}\\
    \theta & = \frac{5}{3}\eta \label{eq:EMM-cond-5}\\
    \psi & = \frac{4}{3}\lambda \label{eq:EMM-cond-6}
\end{align}

Substituting \ref{eq:EMM-cond-4}, \ref{eq:EMM-cond-5}, \ref{eq:EMM-cond-6} into \ref{eq:EMM-cond-1}, \ref{eq:EMM-cond-3} and \ref{eq:EMM-cond-2} gives (\hl{Pepar, please check})
\begin{align*}
    3\gamma + \frac{8}{3}\eta + \frac{7}{3}\lambda & = 1 \\
    6\gamma + \frac{8}{3}\eta - \frac{14}{3}\lambda & = 1 \\
    \implies -3\gamma + 7\lambda & = 0 \\
    \implies \lambda & = \frac{3}{7}\gamma \\
    \implies \psi & = \frac{4}{7}\gamma
\end{align*}
Then
\begin{align*}
    3\gamma+\frac{3}{8}\eta + \frac{3}{7}\gamma + \frac{4}{7}\gamma & = 1 \\
    \frac{8}{3}\eta & = 1 - 4\gamma \\
    \implies \eta & = \frac{3(1-4)}{8}\gamma \\
    \implies \theta & = \frac{5-20\gamma}{8}
\end{align*}
The set of equivalent Martingale measures is therefore:
\begin{align*}
    \mathcal{P} & =\left\{ \left(2\gamma,0,\gamma,\frac{5-20\gamma}{8},0,\frac{3-12\gamma}{8},\frac{4}{7}\gamma,0,\frac{3}{7}\gamma\right): 0\ge\gamma\ge\frac{1}{4}\right\}
\end{align*}

\hl{I copied the condition for $\gamma$ from the notes but this is unphysical. Could you please double check this? }

\section{Part B - Arbitrage-free conditions and market completness}
For this section of the task, we will apply the arbitrage-free conditions to the numerical trinomial tree construction above and also to a completely general trinomial tree. In the later case, we will also derive conditions under which the market is complete. We begin by definining a market $((\Omega, \mathcal{F},\mathbb{F},\mathbb{P}),X)$ with $T=2$ and $d=1$. $X = (1,X_t)$ is comprised of one risk-free asset (normalised to 1) and one risky asset which we will denote as $X_t$. Since we are trying to describe the trinomial model, we know that the sample space will consist of the three possibilities for asset evolution at any given time, namely up $u$, down $d$ and nothing $n$. Assuming the completely general case where the asset price can move a different amount at each time step, we have $ud\neq du$ (commutativity property does not hold) and so $\Omega = \{uu,un,ud,nu,nn,nd,du,dn,dd\}$. For $\mathcal{F}$, we use the natural filtration which prevents look ahead bias. Specifically, $\mathbb{F} = \{\mathcal{F}_0, \mathcal{F}_1, \mathcal{F}_2\}$ where 
\begin{align*}
    \mathcal{F}_0 &= \{\emptyset,\Omega\} \\
    \mathcal{F}_1 &= \sigma(\{\{uu,un,ud\},\{nu,nn,nd\},\{du,dn,dd\}\}) \\
    \mathcal{F}_2 &= 2^\Omega
\end{align*}

We can define the evolution of the risky asset as:

\begin{align*}
    X_{t+1} &= X_tZ_{t+1},\medspace t:0< t\le 2\\
    X_0 & = c\in\mathbb{R}^+\neq0
\end{align*}

with $Z_t,\medspace t:1\le t\le 2$ a sequence of random variable taking on three values: 

\begin{align*}
Z_t(w) &= \begin{cases}
    U_t, \medspace \omega=u \\
    N_t, \medspace \omega=n \\
    D_t, \medspace \omega=d \\
\end{cases}
\end{align*}

In this case, $U_t$ denotes the magnitude of the stock price change at time $t$ and the corresponding definitions hold for $N_t$ and $D_t$. Note that the condition of order is imposed on the values of $Z_t$: $0<D_t<N_t<U_t\medspace$ for all $t$. Finally, the probability measure is defined below with the following constraints enforced (these arise from the definition of a probability measure): $0<p_u,p_d<1$ and $0<p_u+p_d<1$

\begin{align*}
    \mathbb{P} = p_u\delta_u + p_d\delta_d + (1-p_u-p_d)\delta_n
\end{align*}

For the purpose of this assignment, we will assume that the coefficients of $\mathbb{P}$ do not vary with time. Below is an enumeration of all possible values of $X_t$ from $0\let\le2$:

\begin{center}
\begin{tabular}{|c|c|c|c|c|c|}
\hline 
$\omega$ & $Z_1(w)$ & $Z_2(w)$ & $X_0(w)$ & $X_1(w)$ & $X_2(w)$ \\
\hline 
\hline 
$uu$ & $U_1$ & $U_2$ & $c$ & $cU_1$ & $cU_1U_2$ \\\hline
$un$ & $U_1$ & $N_2$ & $c$ & $cU_1$ & $cU_1N_2$ \\\hline
$ud$ & $U_1$ & $D_2$ & $c$ & $cU_1$ & $cU_1D_2$ \\\hline
$nu$ & $N_1$ & $U_2$ & $c$ & $cN_1$ & $cN_1U_2$ \\\hline
$nn$ & $N_1$ & $N_2$ & $c$ & $cN_1$ & $cN_1N_2$ \\\hline
$nd$ & $N_1$ & $D_2$ & $c$ & $cN_1$ & $cN_1D_2$ \\\hline
$du$ & $D_1$ & $U_2$ & $c$ & $cD_1$ & $cD_1U_2$ \\\hline
$dn$ & $D_1$ & $N_2$ & $c$ & $cD_1$ & $cD_1N_2$ \\\hline
$dd$ & $D_1$ & $D_2$ & $c$ & $cD_1$ & $cD_1D_2$ \\\hline
\end{tabular}
\end{center}

At $t=1$, we can write the Martingale condition as follows: 
\begin{align}
    p_u\times cU_1 + p_d\times cD_1 + (1-p_u-p_d)\times cN_1 & = c \label{eq:martin-1} \\
    \implies p_uU_1 + p_dD_1 + (1-p_u-p_d)N_1 & = 1 \label{eq:martin-simplify-1}
\end{align}
assuming that $c\neq0$. At $t=2$, we can write three equations representing the Martingale requirement:
\begin{align}
    p_u\times cU_1U_2 + p_d\times cU_1D_2 + (1-p_u-p_d)\times cU_1N_2 & = cU_1 \label{eq:martin-2} \\
    \implies p_uU_2 + p_dD_2 + (1-p_u-p_d)N_2 & = 1 \label{eq:martin-simplify-2}
\end{align}
\begin{align}
    p_u\times cN_1U_2 + p_d\times cN_1D_2 + (1-p_u-p_d)\times cN_1N_2 & = cN_1 \label{eq:martin-3} \\
    \implies p_uU_2 + p_dD_2 + (1-p_u-p_d)N_2 & = 1 \nonumber
\end{align}
\begin{align}
    p_u\times cD_1U_2 + p_d\times cD_1D_2 + (1-p_u-p_d)\times cD_1N_2 & = cN_1 \label{eq:martin-4} \\
    \implies p_uU_2 + p_dD_2 + (1-p_u-p_d)N_2 & = 1 \nonumber
\end{align}

assuming that $U_1,N_1,D_1\neq0$. Note that equations \ref{eq:martin-2}, \ref{eq:martin-3} and \ref{eq:martin-4} all simplify to \ref{eq:martin-simplify-2} and therefore equations \ref{eq:martin-simplify-1}, \ref{eq:martin-simplify-2} and $\{c,U_t,N_t,D_t\neq0\}$ are the constraints for our trinomial tree. 

In order to determine the conditions for which this market is arbitrage-free / complete, we use the Fundamental Theorem of Asset Pricing (FTAP) 1 and 2. Therefore, we must determine an equivalent Martingale measure and to do this, we construct it arbitrarily as follows:

\begin{align}
    \mathbb{P}^* & = \alpha\delta_u + \beta\delta_d + \gamma\delta_n \label{eq:pstar}
\end{align}

For \ref{eq:pstar} to be an EMM, it must satisfy two conditions: 1) it must be a probability measure and equivalent to $\mathbb{P}$ and 2) $X_t$ must be an $(\mathbb{F},\mathbb{P}^*)$-Martingale. The first condition is confirmed by recognising that both $\mathbb{P}$ and $\mathbb{P}^*$ have the same null set (the empty set $\emptyset$) as well as enforcing the following condition:
\begin{align}
    \alpha + \beta + \gamma &= 1 \label{eq:pstar_cond_1} \\
    \alpha,\beta,\gamma>0 \nonumber
\end{align}
The second condition is satisfied when the following constraints are observed (see \ref{eq:martin-simplify-1} and \ref{eq:martin-simplify-2} except $\{p_u,p_d,1-p_u-p_d\}$ have been replaced by $\{\alpha,\beta,\gamma\}$): 
\begin{align}
    \alpha U_1 + \beta D_1 + \gamma N_1 &= 1 \label{eq:pstar_cond_2} \\
    \alpha U_2 + \beta D_2 + \gamma N_2 &= 1 \label{eq:pstar_cond_3}
\end{align}
Then equations \ref{eq:pstar_cond_1}, \ref{eq:pstar_cond_2} and \ref{eq:pstar_cond_3} form a system of linear equations. 
\begin{align}
    Ax &= b \nonumber\\ 
\left[\begin{array}{cccc} \label{eq:pstar_cond_4}
U_1 & D_1 & N_1\\
U_2 & D_2 & N_2\\
1 & 1 & 1 
\end{array} \right]
\left[\begin{array}{c}
\alpha\\
\beta\\
\gamma\\
\end{array} \right] &= \left[\begin{array}{c}
1\\
1\\
1
\end{array} \right]
\end{align}
Recall that the FTAPI dictates that there are no arbitrage opportunities if and only if $|\mathcal{P}|\neq0$ and FTAPII states that the market is complete if and only if $|\mathcal{P}|=1$. This is equivalent to determining under which conditions \ref{eq:pstar_cond_4} has at least one solution and what conditions does it have only one solution. If $Z_t$ is time invariant i.e. $U_s=U_t$ for all $s,t$ (similar requirement for $D_t$ and $N_t$), we automatically have that the market is arbitrage free since the matrix $A$ contains two equivalent rows which results in an infinite number of solutions. This set of solutions will depend on a parameter (let's denote it $p$) and therefore to enforce market completeness, all that is needed is to select one value of $p$ within its allowable domain. Since symbolic calculations involving \ref{eq:pstar_cond_4} could become tedious, we have instead taken the numerical examples in part A to show an example of this. Recall that $X_t$ evolves according to the table below:

\begin{center}
\begin{tabular}{|c|c|c|c|c|c|}
\hline 
$\omega$ & $X_0(w)$ & $X_1(w)$ & $X_2(w)$ \\
\hline 
\hline 
$uu$ & 1 & 2 & 6 \\\hline
$uu$ & 1 & 2 & 2 \\\hline
$uu$ & 1 & 2 & -6 \\\hline
$uu$ & 1 & 1 & 4 \\\hline
$uu$ & 1 & 1 & 1 \\\hline
$uu$ & 1 & 1 & -4 \\\hline
$uu$ & 1 & -2 & -10 \\\hline
$uu$ & 1 & -2 & -2 \\\hline
$uu$ & 1 & -2 & 10 \\\hline
\end{tabular}
\end{center}
and so equation \ref{eq:pstar_cond_4} becomes:
\begin{align} 
\left[\begin{array}{cccc}
2 & -2 & 1\\
2 & -2 & 1\\
1 & 1 & 1 
\end{array} \right]
\left[\begin{array}{c}
\alpha\\
\beta\\
\gamma\\
\end{array} \right] &= \left[\begin{array}{c}
1\\
1\\
1
\end{array} \right]
\end{align}
assuming that:
\begin{align*}
Z_t(w) &= \begin{cases}
    2, \medspace \omega=u \\
    1, \medspace \omega=n \\
    -2, \medspace \omega=d \\
\end{cases}
\end{align*}
the probabilities become: 
\begin{align} 
\left[\begin{array}{c}
\alpha\\
\beta\\
\gamma\\
\end{array} \right] &= \left[\begin{array}{c}
3/4\\
1/4\\
0
\end{array} \right]
\end{align}
\end{document}
